\documentclass[letterpaper,twocolumn,10pt]{article}
\usepackage{epsfig,endnotes}

\usepackage{adjustbox}
\usepackage{algpseudocode}
\usepackage{algorithm}
\usepackage{amsfonts}
\usepackage{amsmath}
\usepackage{amssymb}
\usepackage{array}
\usepackage[english]{babel}
\usepackage{balance}  % for  \balance command ON LAST PAGE  (only there!)
\usepackage{caption}
\usepackage{color}
\usepackage{comment}
\usepackage{datetime}
\usepackage{paralist}
\usepackage{epigraph}
\usepackage{eurosym}
\usepackage{fancyvrb}
\usepackage[draft,inline,nomargin,index]{fixme}
\usepackage{float}
\usepackage{flushend}
\usepackage[T1]{fontenc}
\usepackage{graphicx}
\usepackage{hyperref}
\usepackage{ifthen}
\usepackage[utf8x]{inputenc}
\usepackage{listings}
\usepackage{lmodern}
\usepackage{makeidx}
\usepackage{marginnote}
\usepackage{mathpartir}
\usepackage{mathptmx}
\usepackage{mathtools}
\usepackage{mfirstuc}
\usepackage{multicol}
\usepackage{multirow}
\usepackage{parcolumns}
\usepackage{paralist}
\usepackage{pifont}
\usepackage{stmaryrd}
\usepackage{subfig}
\usepackage{tikz}
\usepackage[colorinlistoftodos]{todonotes}
\usepackage{url}
\usepackage{verbatim}
\usepackage{xcolor}
\usepackage{xifthen}
\usepackage{xspace}

\graphicspath{{./images/}}
\DeclareGraphicsExtensions{.pdf}

\begin{document}

%don't want date printed
\date{}

\title{GSB Survival Guide}

\maketitle

\section*{About}
Originally this guide was targeted at new Purdue Computer Science graduate students. Since its creation, it has expanded to also be helpful for returning graduate students and perspective students. The guide provides information about the department, Purdue, the Lafayette area, Indiana, and surrounding states.

Be warned that the information in this guide is not complete and not guaranteed to be correct. If you are not sure about something, do not be afraid to ask a fellow student or to e-mail the CS Graduate Student Board (gsb@cs.purdue.edu). If you would like something added or changed in the Guide, please e-mail the GSB.

\textbf{This document is not intended to describe departmental and school policies and is not a publication of the Department of Computer Sciences nor the School of Science.} As the regulations constantly change, it is always best to check with the Graduate School Office (YONG 170), the Computer Science Graduate Office (LWSN 1137), and the Dean of Students Office (SCHL 230).


\section*{History)

This guide has a long and distinguished history, dating back to 1979 when a CS Graduate Guide was founded by Dave Schrader, Eric Dittert, and Bob Brown. Through hard research, diligent work, and ideas stolen from similar guides at other schools, they created an impressively useful document for generations of new students. For a time, it was shared with the EE department, who made additional improvements. Since then, the Graduate Guide has survived countless ordeals, including conversion from the old CDC machines to VAXen and Sequents, departments moving to new buildings, major surgery (both additions and removals) to all sections, and finally, in 1993, conversion to LaTeX. In the Summer of 2007, the guide was ported to the GSB wiki.

The list of people who have helped over the years is far too long to list in its entirety, but includes many students, professors, and department secretaries. Some of the people who have contributed the most in recent years include:

\begin{itemize}
  \item Nate Andrysco
  \item Ethan Blanton
  \item Marina Blanton
  \item Abhinav Jain
  \item Chris Mayfield
  \item Paul Rosen
  \item Andy Scharlott
  \item Abhilasha Spantzel
  \item Lukasz Ziarek
 \end{itemize}
 
 
\section*{Contents}

% ToDo: Add links to the following

\subsection*{The Basics}
\begin{itemize}
  \item \href{https://www.cs.purdue.edu/gsb/doku.php?id=survival_guide:new_students}{For New Students}
  \item \href{https://www.cs.purdue.edu/gsb/doku.php?id=survival_guide:resources}{Resources}
 \end{itemize}

\subsection*{The Department}
\begin{itemize}
  \item \href{https://www.cs.purdue.edu/gsb/doku.php?id=survival_guide:about_the_department}{About}
  \item \href{https://www.cs.purdue.edu/gsb/doku.php?id=survival_guide:curriculum}{Curriculum}
  \item \href{https://www.cs.purdue.edu/gsb/doku.php?id=survival_guide:funding}{Funding}
  \item \href{https://www.cs.purdue.edu/gsb/doku.php?id=survival_guide:computer_access}{Computer Access}
  \item \href{https://www.cs.purdue.edu/gsb/doku.php?id=survival_guide:student_orgs}{Organizations}
  \item \href{https://www.cs.purdue.edu/gsb/doku.php?id=survival_guide:department_events}{Events}
\end{itemize}

\subsection*{The University}
\begin{itemize}
  \item \href{https://www.cs.purdue.edu/gsb/doku.php?id=survival_guide:about_the_university}{About}
  \item \href{https://www.cs.purdue.edu/gsb/doku.php?id=survival_guide:university_buildings}{Buildings}
  \item \href{https://www.cs.purdue.edu/gsb/doku.php?id=survival_guide:university_food}{Food on Campus}
  \item \href{https://www.cs.purdue.edu/gsb/doku.php?id=survival_guide:university_events}{Events}
 \end{itemize}

\subsection*{The Greater Lafayette Area}
\begin{itemize}
  \item \href{https://www.cs.purdue.edu/gsb/doku.php?id=survival_guide:about_lafayette}{About}
  \item \href{https://www.cs.purdue.edu/gsb/doku.php?id=survival_guide:lafayette_attractions}{Nearby Attractions}
  \item \href{https://www.cs.purdue.edu/gsb/doku.php?id=survival_guide:lafayette_shopping}{Shopping}
  \item \href{https://www.cs.purdue.edu/gsb/doku.php?id=survival_guide:bars}{Bars}
  \item \href{https://www.cs.purdue.edu/gsb/doku.php?id=survival_guide:lafayette_newspapers_and_libraries}{Libraries and Newspapers}
  \item \href{https://www.cs.purdue.edu/gsb/doku.php?id=survival_guide:lafayette_recreation}{Sports and Recreation}
\end{itemize}

\subsection*{Leaving Lafayette}
\begin{itemize}
  \item \href{https://www.cs.purdue.edu/gsb/doku.php?id=survival_guide:about_indiana}{Indiana Basics}
  \item \href{https://www.cs.purdue.edu/gsb/doku.php?id=survival_guide:indiana_outdoor_adventures}{The Great Outdoors}
  \item \href{https://www.cs.purdue.edu/gsb/doku.php?id=survival_guide:indianapolis}{Indianapolis}
  \item \href{https://www.cs.purdue.edu/gsb/doku.php?id=survival_guide:chicago}{Chicago}
  \item \href{https://www.cs.purdue.edu/gsb/doku.php?id=survival_guide:other_places_of_interest}{Other Places of Interest}
\end{itemize}

\end{document}







