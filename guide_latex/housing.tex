\section{Housing}

If you do not have any housing by the first week of the semester, run, do not walk, to the Dean of Students Office in Schleman Hall to obtain the Off Campus Housing listing and advice on obtaining a place to live. This information can also be accessed online via:

\centerline{\url{https://www.purdue.edu/offcampushousing}}
\vspace{\baselineskip}

Also check the Exponent:

\centerline{\url{https://www.purdueexponent.org/classifieds}}

and Boiler apartments:

\centerline{\url{https://www.boilerapartments.com/}}

for housing ads and roommate classifieds.
\vspace{\baselineskip}

Grad students often live in one of the Grad Houses or in Purdue Village. If you wish to live in Purdue Village (PV), you should apply ASAP. Purdue Village, which used to be only for married students, does allow single students. Spots in PV tend to fill up fast.

There are a numerous student apartment complexes all around campus and many old houses that have been divided into multiple living units. The apartments right around campus tend to be leased in January and February for the following fall semester, so start your search early in the spring for your fall housing. In addition, if you have a group of friends that you can live with, you can usually find an older house for rent if you check the classifieds. One other resource available to grad students is the Purdue Research Foundation (PRF), which has many old houses around campus for rent. Unfortunately for undergrads, PRF will only rent to faculty and grad students.

Apartments within walking distance of campus tend to be quite expensive but if you have transportation, there are numerous apartment complexes all over the Lafayette area that are quite reasonable. If you don't have a car, you can see if the bus line runs nearby. Of course, you always run a risk if you depend heavily on the buses. One more thing to consider when deciding on off-campus housing is related to restrictions on obtaining parking permits. The University will not sell you a parking permit if you live too close to campus. If you plan on driving to campus, make sure you live far enough away to get a university parking permit.

