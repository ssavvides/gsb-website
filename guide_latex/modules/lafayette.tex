
\subsection{General Description}
The greater Lafayette area has a population of over 85,000, with approximately one-third of that number residing on this side of the river. This population increases about 30 percent when Purdue is in session. The Wabash River separates the cities of West Lafayette and Lafayette. West Lafayette and Lafayette are two distinct cities connected by a number of bridges. If a resident hears you say that West Lafayette is a part of Lafayette or that Purdue is in Lafayette you should immediately fear for your well-being. Remember, Purdue is in West Lafayette.

Lafayette was founded in 1825 by William Digby who named the town after French General Marquis de Lafayette on Lafayette's visit to America that year. Lafayette became a thriving commercial center because of its location on the Wabash River and its proximity to the Old Wabash and Erie Canal. West Lafayette was created in 1866 with the name of Chauncey when the town of Chauncey merged with the town north of it, Kingston. Plans for Purdue were set in 1869 and classes began in 1874. In 1888, Chauncey was renamed West Lafayette.

Purdue is the largest employer in Tippecanoe County. (Tippecanoe is an Indian name for buffalo fish.) Many other industries have been attracted to this area including Alcoa Aluminum, Duncan Electric, Eli Lilly Pharmaceuticals, Fairfield Manufacturing, Landis \& Gyr, National Homes, Ralston Purina, Great Lakes Chemical, Subaru-Isuzu, and Caterpillar Tractor.




\subsection{Weather}
The weather in Lafayette is a constant source of conversation. Summers tend to be hot and winters tend to be cold but the weather is never predictable. One day you may be wearing shorts, and the next day you may be bundled in your warmest winter clothes. Most winters include snow with heavy snowfall at times. The wind-chill factor will occasionally drive the temperature down to or below (Celsius or Fahrenheit, at it really doesn't matter). Spring weather is windy with wide temperature ranges. Some days are very pleasant and others very rainy. Summers bring big thunderstorms and high humidity. Indiana has one of the highest tornado-hit rates in the nation, and Tippecanoe County enjoys the distinction of having the most tornadoes of any county in Indiana. Sometimes flooding occurs in the low-lying areas near the Wabash. The city golf course, many cornfields, and occasionally roads are flooded for weeks at a time. Fall is perhaps the nicest time of the year, with two months of perfect weather for football games and camping.

Weather forecasts for Lafayette are invariably wrong, especially in the Spring when the winds cause weather changes by the hour, but if you insist on calling, the number is 447-0550. Online weather information is also available at http://weather.unisys.com, which uses WXP, a Weather Processor originally developed by the Earth and Atmospheric Sciences Department at Purdue.

\begin{table}[H]
	\centering
	\begin{tabular}{@{}lll@{}}
		\toprule
		Summer Temperature High & 28º C & 83º F \\
		Summer Temperature Low & 16º C & 61º F \\
		Winter Temperature High & 2º C & 35º F \\
		Winter Temperature Low & -8º C & 18º F \\
		Record Temperature High & 41º C & 106º F \\
		Record Temperature Low & -36º C & -33º F \\
		Average Rainfall & 97 cm & 38” \\
		Average Snowfall & 56 cm & 22” \\
		\bottomrule
	\end{tabular}
\end{table}



\subsection{Statistics}
Note that these numbers are somewhat outdated since the greater Lafayette area has grown quite a bit since the last census (the national census figures are collected once every 10 years).

Populations (2010 census figures)

\begin{table}[H]
	\centering
	\begin{tabular}{@{}ll@{}}
		\toprule
		Lafayette & 67,140 \\
		West Lafayette (without students & 29,596 \\
		Tippecanoe County & 172,780 \\
		\bottomrule
	\end{tabular}
\end{table}


