\section{Ph.D.}

The basic requirements for getting a Ph.D. at Purdue are fairly straightforward, as described here: \url{https://www.cs.purdue.edu/graduate/curriculum/doctoral.html}. This section focuses on giving some advise beyond these basic requirements.

\subsection{Advisor}
Your advisor will be the person overseeing your research while you work on your dissertation. In other words, a thesis advisor is a combination of a friend, co-worker, guru, and mentor figure. He or she will therefore be one of the more important people in your life for the next few of years, so choose carefully. Desirable traits in an advisor include:

\begin{itemize}
	\item Easy for you to get along with
	\item Interested in the same area(s) you are
	\item Will not be leaving in the next couple of years (that you can tell)
	\item Can supervise your work closely (if you like that)
	\item Won't pressure you (if you want it that way)
	\item (Optional) Has grant money to support you
\end{itemize}

Usually, you talk to several professors in your area before making a decision. It is possible to change advisors after making your decision, but it is not generally recommended because it tends to delay your graduation.

\subsection{Plan of Study}
Once you have an advisor, your next job is to form the rest of your advisory committee. These will be the people who read your thesis, point out flaws, and eventually decide whether you have done Ph.D.-caliber work. Therefore, they are important people in your education. You and your advisor find (at least) two other professors interested in your area to be on this committee, one of which should be a senior faculty member. About the time you are doing this, you should also file a Plan of Study, an official document telling the administration what classes you have taken, what courses you plan to take, your area of interest, and other vital information. This can be done via \url{https://mypurdue.purdue.edu}.

\subsection{Thesis}
Now that you've demonstrated your aptitude at passing hard tests, and thus qualified yourself for research work, you have to thrash about, reading landmark papers from your area, trying to find a thesis topic. Remember that your goal at this point is to find a topic that you can learn to do research on; that's what the degree process is about. The topic doesn't have to be earth-shattering; in fact, you'll probably get out much more quickly if it isn't. Save the good stuff for when you're on your own trying to get grants and such. Also, consider that by the time you get done with your thesis, you will be eating, sleeping, living and breathing your topic. Try to pick something that you can survive becoming incredibly intimate with for 12 to 24 months; also, by the time you're done, you'll probably be burned out on the topic, so pick something you won't regret not working on for some time after you've graduated.

Once you've figured out exactly what it is that you're going to research, take your Preliminary Examination (usually known as Prelims). The party line on this exam is that it tests the student's competence in a research area and readiness for research on some specific problem. In practice, it is a public thesis proposal, given so that your committee can see what you've been up to, where you're headed, and give constructive criticism. The Graduate Committee will appoint one extra member to your advisory committee for this exam, presumably to keep everyone honest. Usually, this exam is given after you get your first results (publication) on your thesis topic.

Now, work like crazy, trying to prove whatever it is that you're trying to prove. Build, measure, tear down, read, build some more, and conclude. Write it all down in a nice form; we'll call that your \emph{dissertation}. Hope no one else is doing exactly the same thing at another university; if they are, and manage to publish their results before you, even by one week, you're probably out of luck, and have to start all over again on a new topic. Get your committee to agree that they like your dissertation. Then you have to make it satisfy the department's rules for Thesis Format, which define what a CS dissertation must look like, dealing with margins, figures, captions, etc.

Finally, schedule a final defense. This is a public oral exam before your committee and anyone else that cares to come; it is where you present what you've done for the past few years. It's also the last chance for people to pick your work apart and point out flaws. Hopefully, your committee will have pointed them out before the defense, so you have all the answers right at your fingertips. If you've done all your work, this should be a breeze.

