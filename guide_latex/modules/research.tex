\section{Research}

Part of the reason that the department is highly-regarded is that the faculty are active in research, publications, and service to the CS community. It would take pages to describe all the current research projects. Therefore, for reference, the department Research page and Annual Reports page contain a summary of current projects:

\centerline{\url{https://www.cs.purdue.edu/research/}}
\vspace{\baselineskip}
\centerline{\url{https://www.cs.purdue.edu/about/annual_reports.html}}
\vspace{\baselineskip}

There is a research project for anyone here. There are research centers and institutes specializing in particular topics, a complete list of which is given at:

\centerline{\url{https://www.cs.purdue.edu/research/centers.html}}
\vspace{\baselineskip}

Most notably, the Center for Education and Research in Information Assurance and Security (CERIAS) is currently viewed as one of the world's leading centers for research and education in areas of information security that are crucial to the protection of critical computing and communication infrastructure. CERIAS is unique among such national centers in its multidisciplinary approach to the problems, ranging from purely technical issues (e.g., intrusion detection, network security, etc) to ethical, legal, educational, communicational, linguistic, and economic issues, and the subtle interactions and dependencies among them. CERIAS evolved from the COAST (Computer Operations, Audit, and Security Technologies) lab in 1999, which was a multiple project computer security research laboratory in Purdue's computer science department. For more information please refer to \url{https://www.cerias.purdue.edu}.

In addition, there are a number of groups that offer research seminars on a weekly basis:

\centerline{\url{https://www.cs.purdue.edu/research/seminars.html}}
\vspace{\baselineskip}