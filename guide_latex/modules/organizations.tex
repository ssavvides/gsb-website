\section{Organizations}

\subsection{ITaP}
ITaP (Information Technology at Purdue) serves the entire university community (excluding administration). This includes Krannert (business school), Computer Sciences, and other divisions of the University. ITaP provides many varieties of computer systems, and administers several public computer labs and computing clusters.

For more in-depth information, check the schedule of ITaP short courses. These courses are taught by ITaP staff members. Usually they are given in the evening to avoid conflicts with classes or other activities. These courses give you a chance to ask specific questions and increase your knowledge about certain topics. Schedules appear in the ITaP Newsletter and are posted on various bulletin boards. For more information about courses, please visit \url{https://training.purdue.edu}



\subsection{GSB}
The Computer Science Graduate Student Board is the liaison between the department administration and its graduate students. The Graduate Student Board is also affiliated with the Purdue Graduate Student Government. GSB organizes technical talks, pizza parties, summer picnics, bowling nights, movie nights, participates in the graduate and undergraduate committees, and the faculty search process. To learn more about the Graduate Student Board, visit \url{https://www.cs.purdue.edu/gsb/}.



\subsection{ACM}
The International Association for Computing Machinery is an international professional and educational organization dedicated to advancing the art, science, engineering, and application of information technology. The local chapter is open to all Purdue students interested in the field of Computer Science. The goal of the local student chapter is to aid and support student academic, professional, and social development.

ACM supports a number of developmental activities as well as social events throughout the year. ACM sponsors the orientation program for graduate students, the Computer Science fall picnic, programming contests, monthly pizza parties, and guest lecturers. ACM also compiles and distributes the Computer Science Resume Book.

Early in the fall semester, ACM invites Computer Science students to submit resumes which are compiled into a book. The Resume Book is distributed to any company willing to donate a nominal sum. Last year over 100 students participated and over 60 companies donated. The Resume Book sale is ACM's main fund raiser and a great way for students to distribute their resumes to potential employers.

To learn more a bout Purdue ACM visit \url{https://acm.cs.purdue.edu/}.



\subsection{CSWN}
The Computer Science Women's Network (CSWN) is an organization at Purdue University consisting of people (both students and staff) who are dedicated to helping women in the field of computer science. The leadership team that organizes most activities is made up of female students who want to reach out and help all of the women in CS.

CSWN organizes different activities meant to encourage young women to meet one another and also learn more about their chosen field of study. These activities range from picnics to technical talks to helping students find tutors if they are needed. Their goal is to encourage women in computer science to stay in the field and prosper. For information, visit CSWN web site \url{https://www.cs.purdue.edu/cswn/}.



\subsection{USB}
The undergraduate student board is the liaison between undergraduate students and the department administration. For more information, visit \url{https://www.cs.purdue.edu/usb/}.





