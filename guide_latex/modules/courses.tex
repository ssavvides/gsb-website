\subsection{Courses}

First, look at the list of courses being offered on the CS Department web site:

\centerline{\url{https://www.cs.purdue.edu/academic-programs/courses}}
\vspace{\baselineskip}

If you are a first-year Master's students, you will face many choices of classes. The choices for a first-year Ph.D. student are somewhat restricted. Talk to second or third year graduate students. The best place to get information about a course and a professor is from someone who has taken the course, and not necessarily your advisor or professors in the department. This is probably the most important step in the registration process.

Most people find it best to select courses so that their workload is balanced among various types of work: reading, programming, theory, mathematics (calculus, real analysis, linear algebra), etc. Taking two heavy programming courses together is a lot of work, three can be suicidal.

There is also the number of course hours to consider. Typical and maximum course loads are shown below. Keep in mind that what is said to be "typical" below may be a lighter or heavier load than what is right for you. If you are a Master's candidate, how much of a rush you are in to complete your degree will also be a factor. Taking the maximum number of credit hours in your first semester, however, is probably a recipe for disaster.

Credit Hours:
\begin{itemize}
	\item fellowship or self-supported 9 - 12 hours typical, 18 hours maximum
	\item quarter-time assistantship 6 - 12 hours typical, 15 hours maximum
	\item half-time assistantship (most TAs) 6 - 9 hours typical, 12 hours maximum
	\item half-time research assistantship (most RAs) less than 18 hours, at least 6 hours thesis work
	\item full-time research assistantship less than 18 hours, at least 12 hours thesis work
\end{itemize}

You can find the requirements for a Master's and Ph.D. students here:

\centerline{\url{https://www.cs.purdue.edu/graduate/curriculum/masters.html}}
\vspace{\baselineskip}

\centerline{\url{https://www.cs.purdue.edu/graduate/curriculum/doctoral.html}}
\vspace{\baselineskip}

A graduate student is classified as a full-time student if he or she is registered for 6 credit hours when funded by an assistantship or 9 credit hours when funded by a fellowship. Master's students need (eventually) to complete 10 three-credit courses, or 8 three-credit courses with a thesis, for their degrees. One of CS 502 or CS 565, one of CS 503 or CS 536, and CS 580 are required; the others are chosen by the student. You should get an idea of the courses you might like to take now, but don't bother trying to work out a schedule more than a semester in advance --- the actual scheduling of courses (regardless of what the course descriptions say) is quite variable. There are also "topics" courses that are offered each semester, some of which you might find interesting. A 590 topics course is directed study for students who wish to undertake individual reading and study on approved topics. A general topics course is worth three credit hours. It usually takes three to four semesters to complete the work for a Masters degree.



\subsection{Courses Descriptions}

This section contains descriptions of CS courses that are offered on the graduate level in our department. It does not include courses offered by other departments (i.e. MATH, EE, STAT, MGMT) that are also available to obtain graduate credit in the M.S. and Ph.D. programs in CS. For transferring credit check with your academic advisor, or with Secretary to the Graduate Office:

\centerline{\url{https://www.cs.purdue.edu/people/staff/index.html}}
\vspace{\baselineskip}

As there are substantial differences among the courses offered in regard to the amount and type of work for assignments, projects, in-class presentations, term papers, and exams, we are presenting a table that shows the major differences among these courses. The info given is mostly drawn from an old survey among graduate students in our department in Spring 1993, although some additions have been made for courses which were not included in the 1993 survey. Although some of the courses have changed over the years, this list will give you a rough idea of the type of workload to expect. However, course contents and workload depend considerably on the professor who teaches the course. The same number of programming assignments for two courses does not necessarily indicate a comparable effort in writing the code. Therefore, nothing presented here should be taken literally, only as an outline. Do not be afraid to talk to the professor who will teach the course and ask him more detailed information. Note that not all courses are offered every semester. Furthermore, it is not our purpose to show you a way to a degree at Purdue with the least possible effort, but to give you the chance to balance your course load for each semester according to your interests and degree program requirements.

The official prerequisites listed on the course pages are not completely accurate in terms of what you really need to succeed in a course. The survey disclosed that unstated prerequisites for nearly every course. It is not absolutely necessary to know these to do well in every course, but knowing them can greatly increase your efficiency. A comment nearly everyone made at some point was: ``Course are hard and require lots of work ... but in the end it's worth it.'' So, you can look forward to a lot of pain during the semester, and a very good feeling afterwards.

\begin{table}[h]
	\centering
	\begin{tabular}{@{}lp{5cm}p{9cm}@{}}
		\toprule
		\textbf{Course} & \textbf{Course Name} & \textbf{Load} \\
		\midrule
		CS 502 & Compiler Design & written(1), program.(5), proj.(1), quizzes(1), midterm, final - heavy programming \\
		CS 503 & Operating Systems & written(1), program.(5), proj.(1), midterm, final - heavy reading, heavy programming \\
		CS 510 & Software Metrics & - moderate reading \\
		CS 514 & Numerical Analysis & written + program.(8), midterm, final - math and programming \\
		CS 515 & Analysis of Linear Systems & - math \\
		CS 520 & Computational Methods & written + program.(10), proj.(2), midterm, final - math, problem solving, big projects \\
		CS 525 & Parallel Computing & written, program, midterm, final \\
		CS 526 & Information Security & written(5), project(3), midterm, final \\
		CS 530 & Intro. To Scientific Visualization & written, program, midterm, final \\
		CS 535 & Computer Graphics & program.(4), midterm - very heavy programming \\
		CS 536 & Computer Networks & written(5), program.(3),midterm, final - reading, heavy programming \\
		CS 541 & Database Systems & written(5), program.(2), midterm, final - reading, light programming \\
		CS 542 & Distributed Database Systems & written(3), proj.(1), midterm, final - reading, light programming \\
		CS 543 & Simulation and Modeling & written(2), program.(6), proj.(1), presentation(1), midterm, final - heavy programming \\
		CS 555 & Cryptography & written(6), proj.(1), midterm, final - moderate reading and problem solving, math \\
		CS 565 & Programming Languages & written + program.(5), proj.(2), midterm, final - heavy reading, theory, projects \\
		CS 580 & Algorithm Design & written(8), midterm, final - theory and problem solving \\
		CS 584 & Theory of Computation & written(10), presentation(1), quizzes(2), midterm, final - theory, participation in class \\
		CS 603 & Advanced Operating Systems & - reading, systems programming \\
		CS 614 & Ordinary Differential Equations & - math \\
		CS 615 & Partial Differential Equations & - math and programming \\
		CS 636 & Internetworking & program.(3), proj.(1), presentation(2), quizzes(2), oral final - heavy programming, participation in class \\
		\bottomrule
	\end{tabular}
\end{table}


