\subsection{About}

As the story goes, when other college teams met the Purdue football squad they were in awe of the size of the Purdue players. Believing that no man of academic ability could be so enormous, rivals were sure that the Purdue team was made up of workers from the old Lafayette Boiler Factory. Hence, Purdue was the victim of many insulting names, one of which was "Boilermakers".

Other tellers of this tale (probably Purdue opponents) state that the team members actually were boilermakers and not students. Do not believe them. Purdue, being a famous agricultural school, attracted many farm boys who were typically large, healthy, and powerful.

Many Boilermaker alumni have distinguished themselves in one way or another. A few of them are Neil Armstrong, Birch Bayh, Earl Butz, Eugene Cernan, Len Dawson, Bob Griese, Durwood Kirby, Chris Schenkel, Orville Reddenbacher, Roger Chaffee, Virgil Grissom, Abe Gibron, John Wooden, Hank Stram, and Herbert Brown.

Sports have always been big at Purdue, gaining the support of students and nearby residents alike. As a member of the Big Ten football league, Purdue went to the Rose Bowl in 1967 and beat Southern California 14-13. In 1978 the Boilermakers went to the Peach Bowl and beat Georgia Tech 41-21. In 1979 they went to the Astro-Bluebonnet Bowl and beat Tennessee 27-22. The 1980 Liberty Bowl saw Purdue squeak by Missouri, 28-25. Then Purdue suffered three straight losing seasons, before an impressive 1984 season, which ended in a 27-24 loss to Virginia in the Peach Bowl. The team then languished in obscurity until the arrival of coach Joe Tiller. Tiller's Boilermakers have now played–and won–the Alamo Bowl two years running, returning pride to the hearts of fans everywhere. As of 1999, the men's basketball team has shared six Big Ten championships in the last 16 seasons. Three of them came from 1994, 1995, and 1996! The 1987-88 season was the sixth straight year the team won 20 or more games and qualified for NCAA tournament action. Not to be left behind, the women's basketball team was national champion in 1998!

The “All-American” Marching Band is one of the largest university bands in the Big 10 and the nation with 320 members. Highlights of that band include the World's Largest Drum (Built in 1921) and the Golden Girl (A tradition since 1954). The Department of Bands also boasts two to four concert bands each semester and three jazz bands, as well as a 100-member symphony orchestra and the university's pep bands.

The present Purdue seal was adopted in 1974. The griffin head sits on a 3-sectioned shield which represents the 3 educational thrusts of Purdue: science, technology, and agriculture. The lines representing the griffin's mane are for the 5 campuses: West Lafayette, IUPU Fort Wayne, North Central, Calumet, and IUPU Indianapolis.

The Purdue Mascot is the Boilermaker Special V, the locomotive which can be seen around campus primarily before home football games.

Purdue is one of 68 land-grant colleges established with the Morrill Act, an act signed by President Abe Lincoln by which the federal government offered to turn over public lands to any state which would use the land to maintain a college for the study of agriculture and the mechanical arts. The Indiana General Assembly accepted $150,000 from John Purdue and $50,000 from Tippecanoe County. In 1874 classes began at Purdue University with 6 instructors and 39 students.

The West Lafayette campus, including housing areas, recreation areas, the airport, and service areas, covers 2,307 acres. Additional lands away from West Lafayette are used for agriculture and recreation.

The Edward C. Elliott Hall of Music (seating 6,077, it is considered the largest and best-equipped theater of any educational institution in the world), the Loeb Playhouse (seating 1,052), the Experimental Theater, the Memorial Union, Stewart Center, Slayter Center, Ross-Ade Stadium (capacity 67,861), and Mackey Arena (seats 14,123) make Purdue a cultural and recreational center for northwestern Indiana.

The Purdue Airport, established in 1930, was the first university-owned airport in the country.

