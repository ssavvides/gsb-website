\subsection{Buildings}

\subsubsection{Lawson Computer Science Building}
Where most faculty, students, and research labs are located. The administrative staff for the CS department is located here. The basement houses the TA offices and computer labs for students. The 3rd floor has a balcony that overlooks University St. and 3rd St.



\subsubsection{Felix Haas Hall}
The main computer science building until Lawson was completed in Fall 2006. Even though other departments have taken over much of the space, a few labs and faculty still remain in this building.



\subsubsection{Recitation Building}
Until a few years ago, the Recitation (REC) building, directly east of MATH, was only of interest if you had a class there. Starting from the Fall of 1999, however, the second floor of REC is home to the Center for Education and Research in Information Assurance and Security. CERIAS grew out of the COAST laboratory in Computer Science, but with its current Center status is able to have a much larger multidisciplinary reach, although it still has very strong connections with CS. Several CS faculty and staff members as well as grad students have offices there.



\subsubsection{Adminstrative Offices}
Schleman and Hovde Halls are the main student services and administration buildings. There are a number of major administrative attractions in these two buildings.

Hovde:
\begin{itemize}
	\item The Registrar's Office
	\item The Bursar's Office
\end{itemize}

Schleman:
\begin{itemize}
	\item The Dean of Students Office
	\item The Admissions Office
	\item Registration Headquarters
	\item International Student Services
	\item Business Office of Student Organizations
\end{itemize}



\subsubsection{Purdue Memorial Union}
The Memorial Union (in memory of Purdue alums killed in wars) was built back when it was stylish and economically feasible to incorporate a good deal of wood in finishing the interior of a building. The varnished woodwork, solid wood tables and chairs, and stone and wood floors are a refreshing change from the plastic, concrete, veneer and linoleum which surrounds you in most places. Also featured are lots of old moldy plaques commemorating people who would otherwise be forgotten, and a 3-D model of the campus (a must for visiting parents).

Functionally, the predominant features of the Union are eating places, meeting places (various ballrooms and lounges), and sleeping places (the “Union Club” hotel rooms for convention attendees, visiting parents, etc.). For details about the eating places, see the section about on-campus dining.



\subsubsection{Stewart Center}
From here it is possible to walk through tunnels and buildings all the way to either Grad House without going outside, as well as to either of the three parking garages, Marsteller Street (across from Hawkins Grad House), Wood Street (across from Young Grad House), or Grant Street (across from the Union). The main attractions of the Stewart Center are:

\begin{itemize}
	\item Fowler Hall, on the first floor, an auditorium equipped for movie screenings
	\item Loeb Playhouse, the Purdue Experimental Theatre
	\item An Art Gallery, off the west foyer on the first floor
	\item HSSE library, on the first floor, see the section on libraries for details
	\item An ITaP Lab, first floor, often very crowded
	\item ITaP Customer Service Center, on the ground floor, room G68, phone 49-44000; a first and single point of contact for support with many ITaP services
	\item Audio-Visual Center, ground floor
	\item Candy Stand, main floor, sells candy to rot your teeth, paperbacks to rot your mind, and practically every magazine you've ever heard of plus hundreds more that you've never heard of
	\item Envision Center, on the ground floor level between Steward Center and PMU
\end{itemize}


\subsubsection{International Student Services}
The Office of International Students and Scholars in located in Schleman Hall. ISS is a division of International Programs and offers many services that are useful to foreign students. ISS is the expert resource for the University in the areas of F-1, J-1, and H-1B rules and regulations. The office, SCHL 136, is open between 8:00 am and 5:00 pm each weekday, and if you are an international student, you will be visiting every now and then. They are nice folks, even though they may appear a bit harried when you first encounter them around orientation time. To learn more about ISS, visit \url{http://www.iss.purdue.edu}.



\subsubsection{PUSH}
The Purdue University Student Health Center (PUSH) is where the folks with white coats, stethoscopes, and benign smiles are simply waiting to have a look at your innards. They provide the health services to full and part-time students and their spouses, and in certain cases to Purdue employees and visitors.

The hours of operation below are for the regular semesters; while the appointment desk is stays on the same schedule during the summer, some other offices have somewhat reduced hours during summer session and between sessions.

The following services are available without charge. Many of the doctors will accept appointments and a walk-in service is always provided during clinic hours. Walk-ins are first-come-first-serve and you should expect a 15 to 40 minute wait.




