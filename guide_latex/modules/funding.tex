\subsection{Funding}

Most people are funded by either a Teaching Assistantship (TA), a Research Assistantship (RA), or a Fellowship. Some people have sources of funding outside of these three types, but it is uncommon. To be considered a "full-time" student, you must register a certain amount of hours depending on your type of funding. TAs and RAs need to have 6 credit hours, while Fellows need 9. Being considered a full-time student has many benefits that include your ability to receive student health insurance, government loans, etc.

Most students enter the department with a Teaching Assistantship. If you are going to be a teaching assistant, you are probably wondering just what your duties will be. Your teaching assignment will probably fit into one of the following three categories:

\begin{itemize}
	\item A recitation instructor teaches recitation sections which normally consist of 20-30 students. The class will also have other lecture sections that are taught by the professor in charge of the course.
	\item A lab instructor teaches lab sections which normally consist of 15-25 students. The class will also have other lecture sections that are taught by the professor in charge of the course.
	\item A grader grades assignments, projects, and possibly exams for a course that is taught by a professor or another TA.
\end{itemize}

It is very rare that a teaching assistant is the sole instructor for a course, but it has happened in the past, for senior Ph.D. students. Teaching assignments are often not finalized until the week before classes begin. If you did not receive your teaching assignment before arriving at Purdue, talk to the graduate office. Once you have learned your assignment, contact the supervisor of the course as soon as possible. Also, all new teaching assistants must attend ``training sessions'' during the week before classes. These sessions will explain nearly everything you need to know about being a TA.

As a TA, you will be responsible for holding office hours, usually at least two hours a week. If your office hours schedule looks like a typical class schedule (e.g., MWF 1:30-2:30), you risk shutting out students who happen to have a class in that slot. It is much better to make your office hours schedule somewhat irregular. If you are financially supported by the department (TA, RA, grader) and need supplies for your work, they can be obtained in the mail room, LWSN 1151. The secretaries maintain a supply of paper, transparencies, manila folders, tape, pens, and pencils for instructors' and researchers' use.

Depending on your temperament, teaching can either be great fun or a terrible burden. On the positive side, you get paid for the work, you get to meet a lot of new people, and you get to see your students learning and share in their learning process. On the negative side, your students constantly pester you for information and answers, especially before an exam or the due date of a big project. Also, be assured that students will not confine requests for assistance to your office hours. If you have any problems with your assignment, see the course instructor or someone in the graduate office.

Research Assistantships are given to you by a professor who has procured funding from, typically, an outside source such as a government agency (e.g. NSF) or a corporation. Ideally, your RA will support work that interests you and work that will contribute toward your Master's or Ph.D. thesis.

